\specialchapt{ABSTRACT}

Teams have the potential to display high performance or low performance, depending on how well team members interact with one another. Training is commonly used to maintain or enhance the performance of various team types (e.g., sport or work teams). Intelligent Tutoring Systems (ITSs) have been used for years in multiple domains to tutor individuals. However, challenges arise when attempting to develop an Intelligent Team Tutoring System (ITTS). This current work will focus on the challenge of delivering effective automated feedback to teams via an ITTS designed to improve team performance.

The specific goal of this research is to examine how feedback displaying individual and team performance metrics influences team performance across multiple factors. The participants in this study performed a modified version of a classic shopping mall task known as the Multiple Errands Test (MET). The researcher created a three-person team version of the MET called the Team Multiple Errands Test (TMET) within a virtual world on desktop computers. In certain conditions, teams received performance feedback with information about individual performance, team performance, or both. Dependent variables included: performance (individual and team score), items collected (correct and incorrect), errors, time remaining, communication, and task strategy. I divided the results into team and individual level analysis. The preliminary team level analysis suggested that Team level feedback improved both time remaining for the team and time remaining for individuals, but that feedback condition did not affect other dependent measures.  The final dissertation will explore how feedback influences team behavior, such as strategy and communication, as well as team orientation.