% Chapter 1 of the Thesis Template File
\chapter{Introduction}

Teams can achieve more than an individual alone. Teams win sports championships, design and build aircraft that safely transport hundreds of people across the world, and design and build spacecraft that can explore the unknown depths of deep space. In a world that is becoming increasingly complex, it is essential that team members interact effectively with one another. One way to support team interaction is to implement effective team training that improves both team and task skills. Training can come from a human instructor, or when possible, from software that attempts to mimic a human instructor. While Intelligent Tutoring Systems (ITS) have successfully instructed individual students via automated software \citep{Aleven2006, Graesser2018, Hategekimana2008,Koedinger2004}, there are few successful examples of Intelligent Team Tutoring Systems (ITTSs), i.e., software designed to coach teams. A barrier to developing an effective ITTS is understanding the best method of giving feedback to team members. While there are multiple dimensions of feedback, three main dimensions of feedback are Assessment (Team Score Vs. Individual Score), Audience (“Player A, you…” Vs. “Team, you…”), and Privacy (Public Vs. Private). This study focuses primarily on the Assessment dimension. The purpose of this study is to provide insights into the feedback component of team training. The following section explores this challenge in more detail. 

\section{Introduction}

Here initial concepts and conditions are explained and
several hypothesis are mentioned in brief.

\subsection{Hypothesis}

Here one particular hypothesis is explained in depth
and is examined in the light of current literature.

\subsubsection{Parts of the hypothesis}

Here one particular part of the hypothesis that is 
currently being explained is examined and particular
elements of that part are given careful scrutiny.

% Below \subsubsection
% Sectional commands: \paragraph and \subparagraph may also be used

\subsection{Second Hypothesis}

Here one particular hypothesis is explained in depth
and is examined in the light of current literature.

\subsubsection{Parts of the second hypothesis}

Here one particular part of the hypothesis that is 
currently being explained is examined and particular
elements of that part are given careful scrutiny.

\section{Criteria Review}

Here certain criteria are explained thus eventually
leading to a foregone conclusion.



